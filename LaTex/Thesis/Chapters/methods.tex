\chapter{Methods}
\label{ch:methods}

\section{Site}

The Jena Experiment:

\begin{itemize}
\item	50°55´N,11°35´E; 130 m a.s.l.
\item	Established in 2002
\item	10 hectars
\item	Arable field for 40 years before experiment started (therefore strongly fertilized)
\item	Plots are mowed every June and September
\item	Main experiment has 82 plots, each 20x20m (400m²)
\item	Originally sown species mix of 1,2,4,8,16 or 60 species, divided into four blocks (randomized complete block design) along abiotic gradients (mainly soil sand content)
\item	Part of the Plot is weeded twice a year (not my area)
\item	I collected data in the “old invasion plots” (6.35m x 4.55m , 28,98m²) and in the “new invasion plots” (5.35m x 3.55m, 19m²)
\item	Observations were made only during good weather conditions (max partly overcast, no rain, max light wind, min. 15 degree)
\item	Sampling time between 9am and 5pm (there was normally heavy fog and moist in the mornings so I could only start sampling from 10 or even 11)
\item	Sampling occurred between 20.7. -  12.8.
\end{itemize}

\section{Choosing the Plant & the Plots}





\section{Statistical Analysis}

  
