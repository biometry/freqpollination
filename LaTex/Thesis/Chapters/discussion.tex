\label{ch:discussion}
\section{Discussion}
Frequency dependence can influence the plants fitness and therefore have far reaching consequences for the development and maintenance of biodiversity. Aim of this thesis is to study the existence of frequency dependence in a natural plant community, explore the kind of relationship and identify the factors influencing frequency dependence with the help of an agent-based model.\\
Evidence for frequency dependent pollination was found for rewarding flower species in grassland plant communities. The relationship is defined by a steep increase of visits within the first 20\% frequency followed by a disproportional low gain of visits for every additional flower due to an increase of frequency. When the species becomes dominant the per-flower visitation rate increases again. The combination of negative and positive frequency dependence combined in a cubic curve is generally supported by the agent-based model. Furthermore, the model reveals floral cover and cluster size as important influencing factors for frequency dependence. \\


Previous research found positive frequency dependence for lab experiments on rewarding flowers and inconsistence in the few field experiments focusing on color morphs \citep{smithson2001pollinator}. However, the field data shows a cubic pattern of frequency dependency for at least four different rewarding flower species and the results from the foraging simulation support the general shape of the per-flower visitation ratio. Where does the discrepancy of lab and field data comes from?\\ 

The sensitivity analysis of the agent-based model can give an explanation: If the reward function is increased to a refill within 10 time steps, the relationship is reversed to a positive frequency dependence and the rare species receives disproportionally few visits (Fig.~\ref{fig:SA_reward}). The curve is consistent with findings of  \cite{smithson1997density} and \cite{smithson1996frequency}. In their study design, artificial flowers were refilled after each foraging bout. Therefore, every bumble bee foraged on a set of full and equally rewarding flowers which is comparable to a high regrowth function in the agent-based model.\\
We know that pollinators more likely abandon flower constancy if they experience sequentially bad reward \citep{chittka1997foraging,goulson1994model}. If the reward is always high, pollinators have less incentive to go on exploratory visits to the rare species as the abundant type is easy to find and sufficient rewarding. Hence it can be assumed that negative frequency dependent selection does not exclusively apply for non-rewarding species but also for flowering communities with varying or insufficient reward. Strong positive frequency dependence in pollination might be only possible for highly rewarding or artificial systems. 

%More research, especially on natural field conditions, is needed to confirm this hypothesis.

%high nectar production rates increases approach rates Pleasants 1981; Real & Rathcke 1991;Duffield al 1993; Shykoff & Bucheli 1995 (from  Klinkhamer 2001)

\subsubsection*{Floral cover and cluster size are influencing factors}
The model reveals two main influences for frequency dependence: The higher the floral cover, the stronger the frequency dependence and the bigger the clusters, the weaker the frequency dependence (Fig.~\ref{fig:PFV}e-h). 
Floral density is known to influence visitation rates, usually positive and with a saturating function (eg. \citealt{rathcke1983competition,essenberg2012explaining,bernhardt2008effects,Kunin1997}). Those findings are consistent with the Holling´s type II functional response found for different cover values (Supplementary material, Fig.~\ref{fig:nonfreq}a). If the cover is increasing, the absolute number of flowers rises also for the rare species. That makes it more likely for a bee-agent to find a flower before changing preference towards the common species due to long search times even if foraging on the later would be more efficient. Therefore, high cover causes the same effect as expanded vision distance or maximum search limit (cf. Fig.~\ref{fig:SA_view} and Fig.~\ref{fig:SA_flight}): The main reason of abandoning flower constancy becomes multiple visit of flowers with low reward. Consequence are explanatory visits to the rare species which can have a great impact on the per-flower visitation rate. Every visit to a rare species weights high in the per-flower visitation rate because the sum of visits is divided through the number of flowers.\\ 

The model shows that spatial aggregation of flowers can lead to a more efficient foraging (more visits per time unit), less frequency dependence and a higher quality of visits due to compatible pollen deposits. If flowers are randomly distributed over the whole meadow, many short search and flight times apply. An intermediate cluster level is easy to exploit by a pollinator whereas the flight and search times can be very long in between few big clusters, especially for low floral densities. \\
It was already suggested by \cite{epperson1987frequency} that spatial agglomeration of flowers decreases frequency dependence. In the agent-based model, a similar effect compared to low cover takes place: If flowers are aggregated at few places, they affect the pollinators perception of frequency and are more difficult to find. Long search times will weaken the bee-agents flower preference and lead to foraging on the next available cluster independent of its species. 

%more literature!

\subsubsection*{Requirements for successful pollination}
High visitation rate is gained at low frequency with high cover and low cluster size. However, those visits might not be the best quality if the pollination per visit ratio is comparatively low (Fig.~\ref{fig:POC}a,d). The ratio can be seen as index for flower constancy: If the majority of visits lead even for a small pollen carryover value to successful pollination the bee-agents forage very flower constant \citep{montgomery2009pollen}. If the cover is high, bee-agents will keep their constancy also for rare species because they are abundant enough. If the aggregation of flowers is high, bee-agents exploit this cluster before leaving for the next. Every visit within a cluster of flowers of the same species is counted as successful pollination and can lead to a high visit quality even if the cover is low (cf. \citealt{jakobsson2009relationships}). Therefore, rare species should occur in clusters of flowers if the cover is low to get a maximum of pollination per visit. If the cover is high the spatial distribution plays a minor role for the visit quality.  

%vgl Hanoteux?
%see Feldmann 2008

\subsubsection*{Sum of visits to the flower community also show frequency dependence}
Additionally to individual frequency dependence, I analyzed the impact of species partitioning on frequency dependent visitation in the system as whole. \\
If the cover is very low, a maximum in total visits can be received if one species is dominant. Co-flowering will lead to longer search times and less overall visits (u-shape for 5\% cover in Fig.~\ref{fig:SUM}a). For higher cover, the frequency dependence shows a function similar to a fourth-degree polynomial. If one species is rare at 5-20\% frequency, pollinators exhibit exploratory visits to the rare species and spend inefficient time searching. Hence, the total visitation number drops to a minimum. If species are evenly distributed the pollinators forage on both species in equal amounts. This is the most efficient status for the overall ecosystem, especially for high cover or cluster size. \\
Spatial aggregation weakens the frequency effect but will also reduce the total visits. If flowers are even and random distributed, the bee-agent has many small search times intermittent by collecting reward on a single flower and continue foraging. Rare flowers can be found comparatively easy if they are spread over the whole meadow and flower constancy will be kept even if it is highly inefficient. If the clusters of flowers are bigger, bee-agents will not find rare flowers that easily because they might occur only in a single cluster on the meadow. The bee-agent switches to the common flower, the minimum at very uneven distribution disappears and the relationship becomes slightly hump-shaped (Fig.~\ref{fig:SUM}b,c). \\
A maximum of overall visits favoring both the pollinators and the co-flowering plants can be achieved with balanced abundance in high cover and very uneven species distribution for low floral cover environments. 

\subsubsection*{Limitations of the study design and research suggestions}
Even though modeling can be an excellent tool to understand and interpret ecological data, some questions evolve comparing the data collected in the Jena Experiment and the foraging model. Floral cover is an important factor in the outcome of the model. It influences not only the absolute number of visits but also the intensity of frequency dependence. But it was removed in the model selection as it was no factor of explanatory power to the per-flower visitation data. 
Reason could lie in the sampling design. Data was only observed from plots with an intermediate cover, no extremes were taken into account. In total, there were only five values for cover in the final analysis. Also all cover values are estimations, no exact measurements.\\
The data collected in Jena shows drastic differences in attractiveness of the focal species and frequency dependence was found to be subject to each species. Therefore I suggest research on a variety of species, both rewarding and unrewarding with measurements of reward.\\
Validation if further results of the agent-based model can be made via a supplementary study with varying frequency, cover and cluster size of two co-flowering species. Either under natural conditions where manipulation is possible (eg. \citealt{Eckhart2006frequency,essenberg2012explaining}) or with potted plants \citep{epperson1987frequency}. Also results of the frequency dependent sum of visits could be tested by manipulated field experiments. Natural conditions data are not suitable for this purpose because every plot contains more than two co-flowering species with unequal attractiveness.  
