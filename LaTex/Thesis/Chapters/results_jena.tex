\label{ch:results_jena}

\subsection{Results}

\subsubsection*{Visitation Rates}			

In total, I made 385 observations, each representing pollinator activity records for 15min in a 80cm x 80cm patch representing 96,25h of observation on 246.4m$^{2}$.\\
\textit{Onobrychis viciifolia} was the most attractive plant with a maximum of 318 visits in one observation. The per-flower visitation rate varied strongly with the attractiveness of the focal species. Per observation, I recorded 1.4 $\pm$ 1.8 (mean $\pm$ SD) visits per flower with a maximum of 10.7 visits per flower (again \textit{Onobrychis viciifolia}) and 31 observation with no visit at all to the focal species. The per-flower visitation rate was significantly different between the two very attractive species \textit{Geranium pratense} and \textit{Onobrychis viciifolia} and the three less attractive species  \textit{Trifolium pratense}, \textit{Lotus corniculatus} and \textit{Lathyrus pratensis} (\textit{P} $\leq$ 0.001, Tab.~\ref{tab:VR_spec}). The subplots contained 3 $\pm$ 1.2 (mean $\pm$ SD) flowering species including the focal species, the species richness was higher at the plot level with 8 $\pm$ 2.4 (mean $\pm$ SD) flowering species.

\begin{table}[!htbp] %VR species
	\centering
	\caption{Per-flower visitation rates (mean $\pm$ SD) for the focal flower species per 15 minute observation. \textit{Geranium pratense} and \textit{Onobrychis viciifolia} are significantly different from the other three species (pairwise t-test, \textit{p} $<$ 0.001) . Within the two groups, there is no significant difference between the species. }
	\begin{tabular}{l l l l l}
		\toprule
		\textbf {Short} & \textbf{Species} & \textbf{Family} &\textbf{Visitation Rate (Mean)} & \textbf{SD} \\
		\midrule
		Ger  & \textit{Geranium pratense} & Geraniaceae & 3.05 & 1.5 \\ %109
		Lat  & \textit{Lathyrus pratensis} & Fabaceae & 0.57 & 0.53 \\ %83
		Lot  & \textit{Lotus corniculatus} & Fabaceae & 0.30 & 0.36 \\  % 77
		Ono  & \textit{Onobrychis viciifolia} & Fabaceae & 3.60  &  2.5 \\ % 37
		TP   & \textit{Trifolium pratense} & Fabaceae & 0.16 & 0.23 \\ % 79
		\bottomrule
	\end{tabular}%
	\label{tab:VR_spec}
\end{table}%


\subsubsection*{Frequency Dependence}			
Floral cover and species richness had both individually and in the interaction term with frequency no effect on the visitation rate and were removed from the model (Cover: F$_{df\shorteq1}$ = 1.17, \textit{P} = 0.28; Species Richness: F$ _{df\shorteq1} $ = 1.15, \textit{P} = 0.29). 
%Freq:Cover 0.446 //Freq:SR 0.426

The linear mixed effect model showed an effect of species and frequency individually and with interactions on the per-flower visitation rate (Species: F$_{df\shorteq4}$ = 141.13, \textit{P} $\leq$ 0.0001; Frequency: F$_{df\shorteq1}$ = 18.29, \textit{P} $\leq$ 0.0001; Species x Frequency F$_{ df\shorteq4}$ = 5.2, \textit{P} $\leq$ 0.001, Tab.~\ref{tab:anova}). Interestingly, frequency contributed also as quadratic and cubic term with its interactions to species explanatory power to the model, giving the relationship a non-linear character. Figure~\ref{fig:LME} shows a cubic relationship of four focal species and the summed data and a quadratic relationship for \textit{Lathyrus pratensis}. The cubic curve is defined by a strong increase for frequencies below 20\% followed by a minimum between 50 and 80\% depending on the species before raising again with increasing dominance of the focal species. However, the visitation rate of \textit{Lathyrus pratensis} showed a maximum at 60\% frequency and decreases afterwards. 

\begin{figure} [!th] %LME
	\centering
	\includegraphics[width=14cm]{Images/LME}
	\caption{Relationship between per-flower visitation rate and frequency for the five focal species. Each point represents one observation of 15 minutes. The y-axis is plotted on a log scale due to the wide variance in attractiveness between the focal species. The model shows a cubic frequency dependence for all species but \textit{Lathyrus pratensis}. The dashed line represents the prediction for the overall model with its 95\% confidence-interval (grey area). Floral cover and species richness were dropped as explanatory variables in the model selection process. R$^{2}$ was calculated with the "r.squaredGLMM"-function of the MuMIn-Package \citep{MuMIn}}
	\label{fig:LME}
\end{figure}

\begin{table} [!htbp] %results lme
	\centering
	\caption{Results of the linear mixed effect model with per-flower visitation rate as response variable. Floral cover and species richness and its interactions with frequency were removed in the model selection process (denDF = 191, R$^{2}$ = 0.53, n = 385)}
	\begin{tabular} { l c c c}
		\toprule
		\textbf{Explanatory Variables} & \textbf{Df} & \textbf{F-value} & \textbf{\textit{P}} \\
		\midrule
		Frequency 			&  $1$ & $49.3$ & $<0.0001$ \\
		Frequency$^{2}$ 		&  $1$ & $13.2$ & $0.0026$ \\
		Frequency$^{3}$ 		&  $1$ & $ 5.8$ &  $0.8145$ \\
	    Frequency x Species &  $4$ & $ 5.2$ &  $0.0005$ \\
		Frequency$^{2}$ x Species & $4$ & $3.4$ & $0.0097$\\
		Frequency$^{3}$ x Species & $4$ & $3.4$ & $0.0101$\\
		\bottomrule
	\end{tabular}
	\label{tab:anova}
\end{table}
