\label{ch:abstract}

\section*{Abstract}
%250 words

Flower frequency dependence occurs if the frequencies in the flowering community affect the foraging behavior of   pollinators. By influencing the plants fitness, frequency dependent pollination could have far-reaching consequences for plant coexistence. Negative frequency dependence, hence a pollinators preference for the rare species, is thought to enhance diversity and to be the reason for color morphisms in rewardless orchids. Common species on the other hand benefit from a positive frequency dependence which can reduce diversity. However, only few studies have been conducted on frequency dependence with inconsistent results. Focus in this thesis is the analysis of frequency dependence for rewarding species in natural flower communities and the identification of influencing factors using a combined approach of field and model data.\\
I observed pollinator visitation to flowers of five rewarding species in their natural plant community in the area of the Jena Experiment. Thereupon I used an agent-based model of two co-flowering plant species competing over pollination service by a shared pollinator to identify patterns and influencing factors of frequency dependence. \\
Four out of five species showed a cubic frequency dependence in the field data. Furthermore, the results of the model support the general relationship and identify floral cover, cluster size and reward as important influencing factors. Negative frequency dependence was found for high floral cover, positive for low cover or very high reward.\\
In conclusion, my results indicate the existence of frequency dependent pollination for rewarding species in a natural flower community. Also, frequency dependence appears to depend on floral cover and spatial aggregation of flowers. Therefore, patterns of frequency dependence are likely to change in time and space and are not solely related to reward and certain species. In consequence,, frequency dependence might be more important concept than previously thought for evolution and maintenance of diversity. \\

\vspace{1cm}
\small{{\textit{Keywords:} Pollination, frequency dependency, flower constancy, foraging behavior, coexistence, agent-based model, density, patchy environment, nectar-production rates}}
