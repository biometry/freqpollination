\section{Conclusions}
\label{ch:conclusions}
In conclusion, this study shows for the first time that frequency dependent selection exists in natural flowering communities for rewarding species. Also, a combination of methods is exceptionally helpful to understand influencing factors. The output of the agent-based foraging model confirms the results of the field data and fills the knowledge gap of previous research: Positive frequency dependence proved multiple times in lab experiments is likely due to very high rewards. Negative frequency dependence is therefore not exclusive to non-rewarding morphs but takes effect also for common rewarding species if the reward is not exceptionally high. Patterns of frequency dependence can therefore change across space and time, especially because the model revealed floral cover as FD-increasing and spatial aggregation of flowers FD-reducing factor.

Those findings are important for our understanding of the evolution and conservation of diversity. Negative FD, thus if rare flowers have an advantage in pollinator visits, might be an important factor in the evolution not only floral polymorphisms but diversity as such. 

Further research is necessary to validate the role of floral cover and cluster for FD. A controlled field experiments including measurements of floral reward and pollination success could be a suitable approach. Based on the findings, I also recommend the connection of modeling, field and lab work. Most research is only done in one of those three approaches with a lack of cross-validation. It could be gained a great deal of knowledge by establishing interdisciplinary working groups of field ecologists and environmental modeling experts. \\


What else? (Nor yet included)
\begin{itemize}
	\item POC
	\item Best for ecosystem and bees (most visits, most efficiency): extreme or balanced frequency, high cover, intermediate cluster
\end{itemize}
