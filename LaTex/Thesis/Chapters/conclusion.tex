\section{Conclusions}
\label{ch:conclusions}
%final conclusions, applications, further research
1. "in conclusion, our study shows (what you think is the main message)"


2."Important for our understanding of …"


3."Based on our findings, we recommend …"


4."Further research is necessary to finally establish …"


5. Suggest practical applications of your results?
- More knowledge about pollination always good: Big environmental factor (diodiversity, function of ecosystems, economically)
- Not direct practical applicable
- maybe: how to integrate other flowering plants to enhance pollination in crop production?


Here, the combination of methods is exceptionally helpful to understand underlying drivers. The interpretation of ecological results is often based on guesses. With a combination of methods,an methodically exploration of reasons is possible.

more connection of modeling, field and lab work. Most research is only done in one of those three approaches. Crossvalidation is missing. It can be gained a lot by getting the field-loving outdoor-ecologists and the computer-statistic-geeks working more together. NetLogo is learnable in a reasonable short time, even for non-statistitians. 

Extent model:
\begin{itemize}
	\item integrate metalevel in model/ multiple plots in one world
	\item more species
	\item similarity gradient
\end{itemize}


General outcome: cover increases frequency dependence because also the rare species is more abundant and clustering decreases frequency dependence because if flowers are clustered, they are again more difficult to find. A rare species gets the most visits in a high cover and low cluster environment.

