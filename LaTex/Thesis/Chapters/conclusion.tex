\section{Conclusions}
\label{ch:conclusions}
In conclusion, this study shows for the first time that frequency dependent selection exists in natural flowering communities for rewarding species. Also, a combination of methods is exceptionally helpful to understand influencing factors. Both, the output of the agent-based foraging model and the results of the field data show a cubic influence of frequency on the per-flower visitation rate. More pollinator visits can lead to more seeds and an advantage in the lottery competition in reproduction and therefore influence the fitness of a species. \\
The sensitivity analysis can fill some knowledge gaps in previous research: Positive frequency dependence proved multiple times in lab experiments is likely due to very high rewards and low cover values. Negative frequency dependence is therefore not exclusive to non-rewarding morphs but takes effect also for rewarding species if the reward is not exceptionally high or the cover is very low. \\
Positive and negative frequency dependence may not be object to certain species. The results suggests that it is more likely that patterns change across space and time, especially because the model revealed floral cover as increasing and spatial aggregation of flowers reducing factor. If frequency dependence is in fact found for a variety of rewarding flowers not only floral polymorphisms, I agree with \cite{Eckhart2006frequency} that it might be more important in the evolution and conservation of diversity than previously recognized.\\
Further research is necessary to validate the role of floral cover and cluster for frequency dependence. A controlled field experiment including measurements of floral reward and pollination success could be a suitable approach. Based on the findings, I also recommend the connection of modeling, field and lab work. Most research is only done in one of those three approaches with a lack of cross-validation. A great deal of knowledge could be gained by establishing interdisciplinary working groups of field ecologists and environmental modeling experts. \\