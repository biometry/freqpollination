\label{ch:methods}

\section{Field Study} 
%Natural condition experiments
%natural frequency experiments
%field data

\subsection{Methods}

\subsubsection*{Study Site}
The data used in this analysis were collected in the the Jena Experiment, located north of the city of Jena in the centre of Germany (N\ang{50;55;} E\ang{11;35;} ; 130 m a.s.l.) in July and August 2014. Mean annual temperature is $9.3 ^\circ\text{C}$ and mean annual precipitation 578mm \citep{kluge2000klima}. In 2002, 10ha of strongly fertilized arable field in a floodplain of the Saale river were converted into a biodiversity experiment. Species mixes of 1, 2, 4, 6, 8, 16 and 60 species from a pool of 60 common European grassland species were sown in 82 plots 	\`{a} a 20m x 20m \citep{roscher2004role}.  The Jena Experiment has the purpose to explore the effect of plant diversity (species richness and functional group richness) in grassland communities and is used in numerous studies and experiments.

The plots of the Jena Experiment are mowed twice a year in accord to standard grassland management. Parts of each plot are additionally weeded twice a year to maintain the original plant composition. Two subplots per plot were excluded from the weeding since 2002 ("Old Invasion Plots", 4m x 5.5m , 22m$^{2}$ ) and since 2009 ("New Invasion Plots", 5m x 3.5m, 17.5m$^{2}$) to evaluate invasive potential and effects. In 2014, subplots with continuous weeding were scarce with flowers and had a generally low species richness. Hence I collected the data in the old and new invasion plots with a higher cover, species richness and diversity. From the 82 plots of the Jena Experiment I only included plots with a floral cover between 20\% and 70\% for better comparability. In total, 23 plots were sampled throughout this study. 

\subsubsection*{Data Collection}
I selected the focal plant species during the field work as the flora changed very quickly and unpredictably. A focal species had to be flowering for at least one week in the sampling time and be present in at least five plots with a differing frequency to get sufficient data. Therefore, I chose \textit{Lathyrus pratensis}, \textit{Lotus corniculatus}, \textit{Trifolium pratense} and \textit{Onobrychis viciifolia} of the family Fabaceae and \textit{Geranium pratense} of the family Geraniaceae (Supplementary material, tab.~\ref{tab:Species}).

Pollinator observations were only made during suitable weather conditions (at most partly overcast, no more than light wind, min. $15 ^\circ\text{C}$). The sampling took place between 9am and 5pm. Overall, 15 days between 20th of July and 12th of August 2014 were suitable for pollinator observations. 
During each observation bout I recorded all pollinator activity during 15 minutes in a patch of 80cm x 80cm. This size is feasible to watch even with high pollinator activity and floral cover. The data collection included all visits to flowers of the focal plant species and total visitation number for all other flowers in the patch. I counted the flowers of the focal species to calculate the per-flower visitation rate. As possible drivers for visitation rate changes, I estimated the floral cover and identified all other flowering plant species present on patch and plot level. Each plot contained eight evenly distributed patches for 2h observation time per focal species and frequency. 
%as shown in Figure~\ref{fig:plot-design}

\subsubsection*{Statistical Analysis}
Per-flower visitation rate was used as response variable to identify frequency dependence. The variable equals the count of visits of all pollinator types to all flowers of the focal species per patch within 15 minutes observation time divided trough the number of flowers of the focal species within the patch. Therefore, the response variable is not a count data and was modeled with a Gaussian error distribution in the analysis.
The explanatory variables of the full model were measured at plot level and included species richness, floral cover as single terms and frequency as single, quadratic and cubic term. Included interaction were frequency (single and quadratic) with floral cover and species richness as well as frequency (single, quadratic and cubic) and cover in interaction with species. Species was included as nominal response variable. All statistical analysis were performed with R, version 3.1.2. \citep{R}. \\

%All variables also existed on a subplot level but with only 386 data points, I decided to focus on the plot level.

I used variance inflation factors (VIF) to check whether any variables in the dataset are collinear and should be removed prior to the analysis. With all values below two, there was no sign for collinearity and therefore acceptable to use them in the model selection as explanatory variables (\citealt{zuur2007analysing}, supplementary material, tab.~\ref{tab:VIF}). Pairwise scatterplots with included correlation of coefficients also showed only minor correlation (Supplementary material, fig.~\ref{fig:pairs-plot}).

The sampling design contained 8 observations per plot summing up to 2h of observations per species and frequency. Therefore, the data are not independent and I chose a linear mixed effect model with subplot nested in plot as random effect. I used the function "lme" from the R package "nlme" \citep{Rnlme} for all further analysis.
%Zuur p147

The beyond optimal model with the full set of reasonable predictors and interactions showed a strong pattern of heteroscedasticity in the residuals. With the varIdent-function from the R-package "nlme", every species is allowed to have its own variance structure and we can maintain the differences in attractiveness of the five focal species in the model as biological information. The weighting provided a significantly better variance structure for the model (\textit{L} = 383.74, \textit{df} = 4, \textit{p} $<$ 0.0001).

I performed a backward stepwise deletion of interactions and predictors with maximum likelihood estimation (ML) for each model. The loss of explanatory power in the model after removal of a variable was tested by comparing the Akaike information criterion (AIC) of the model with and without the explanatory variable (ANOVA model comparison). If there was no significant loss of explanatory power, the variable was removed. The selection was verified by a global selection via the dredge-function from the R-package "MuMin" \citep{MuMIn} with maximum likelihood estimation. 

%The mixed effect model was significantly better compared to a generalized least squares model with the same set of explanatory variables (models estimated with REML, \textit{L} =7.97 , \textit{df} =2 , \textit{P} = 0.0186).\\

The final model was again validated by plotting the normalized residuals against fitted values. The vertical gap in the residuals can be explained by the difference in flower attractiveness (supplementary material fig.~\ref{fig:residuals}). \textit{Geranium pratense} and \textit{Onobrychis viciifolia} received very high visitation rates whereas \textit{Trifolium pratense}, \textit{Lotus corniculatus} and \textit{Lathyrus pratensis} had generally only few visits. However, the heteroscedasticity of residuals could be dealt with by the weighting and the mean of the residuals is close to zero ($<$ 0.0001). 