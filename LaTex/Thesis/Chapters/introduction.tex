\section{Introduction}

\subsection*{Storyline}


%
%1. Broad Topc (1-2 Paragraphs), what has benn done before
%2. Introduce Problem (1-2 Paragraphs) Gap in knowledge
%3. My approach (1 paragraph) What method data did I use?
%4. Questions (goes also with point 3)

\begin{itemize}

\item biodiversity loss /important to maintain and understand 
\item what are the divers for biodiversity?
\item pollinator loss, have to understand underlying rules/drivers/

\item pollination crucial for most plants (self-incompatible)
\item Interspecific and intraspecific competition
\item pollinator service shared ressource of coflowering plants
\item increased forager activity and attractiveness can respond in higher fitness

\item response of pollinators to certain variables like density, frequency, traits got a lot of attention over the last 2 decades since pollinator decline
\item complex multidimensional interactions
\item other factors: presence/absence of similar flowering species, spatial scale
\item detects bits and peaces to get a greater understanding
\item The effect on density is rather well studied

Science agrees on a form of density dependence for pollination success from various studies with lab, field and modelling approaches:

\citep{bernhardt2008effects} field
\citep{elliott2009effects} field
\citep{essenberg2012explaining} numerical Model and field
\citep{Kunin1997} field
\citep{kunin1993sex} experimental
\citep{morris2010benefit} model, field
\citep{rands2010effects} model
\citep{stout1998influence} lab

usually positive

%%%%%%%%%%%%%%


Frequency-dependence (FD) of survival or reproduction occurs when the relative fitness of a species changes as a function of its frequency in the community. 

According to ecological theory, frequency dependent pollination could have far-ranging consequences for species coexistence and the maintenance of diversity: While negative frequency dependence (i.e., preferential pollination of rare flower types) is thought to increase diversity, positive frequency dependence (higher pollination success for common flower types) tends to reduce diversity in modeling studies (May 1974, but see \citealt{bever1999dynamics}, \citealt{molofsky2002novel}).

 +FDP
- fixation on one phenotype for color morphs of one species
- possible reduction of diversity (rare flowers do not get enough pollination, reproduction disadvantage) see Molofsky and Bever 2002
- the most frequent species is becomming even more frequent

resons/explanations:
 - search image hypothesis (sensory system becomes trained, minimize search times)
 - search rate hypothesis (tradeoff search time and probability to find the next rewarding flower)
 
 -FDS
 - promotes phenotye diversity
 - can promote landscape diversity
 - known for non-rewarding flowers
 
 reasons:
 - if a species is rare enough, the negative experience is not stored long enough in the short term memory and it gets exploratory vistits by the pollinator


For animal-pollinated plant species, optimal foraging theory predicts that under most circumstances pollinators should favor common flower types over rarer ones \citep{kunin1996pollinator} , but so far this has rarely been tested. 


Research:
Most previous studies of frequency-dependent pollination focused on within-species variation in pollination success of different flower color morphs in the field or in the lab with a "bee-board" with corolla color as differing flower trait. 

In the review of \citealt{smithson2001pollinator} 11 of 13 lab experiments showed a significant FD. Bumblebees tend to visit the common morph if rewarding and the rare morph if unrewarding (for simple and controlled conditions for color morphs). Lab easy o get clear colors and equal rewards for all color morphs.

\citep{smithson1996frequency}
\citep{smithson1997negative}


Not so clear for field experiments. Mostly just trends without significance (for both negative and positive FDS)

\citealt{Eckhart2006frequency}
\citealt{epperson1987frequency}
\citealt{gigord2001negative}, 
\citealt{jones1996pollinator}


In the review by \cite{smithson2001pollinator}, 11 of 13 lab experiments using artificial flowers on a "bee-board" with rewarding flowers showed significant results for frequency dependence. 10 of those were done with rewarding flowers and resulted in positive frequency dependence. The only experiment resulting in negative frequency dependence was done with non-rewarding flowers \citep{smithson1997negative}. The few field experiments on frequency dependence concentrate on color morphs and are not consistent. While \cite{epperson1987frequency} found the rare white morph of \textit{Ipomoea purpurea} to be undervisited (but not the colored morphs), \cite{Eckhart2006frequency} was the first to prove negative frequency dependence for a rewarding species (\textit{C. xantiana ssp. xantiana}). Other experiments had no significant results (eg. \citealt{jones1996pollinator}) and experiments on natural flower communities lack completely to our knowledge. 


rarely been addressed: 
FD of certain species within a floral community (in contrast to flower morphs)

been considered in the context of frequency dependence.


new because: 
-never been a comparison model and field data for fd
-non-linear relationship (consistent)
-findings in the field
- most models see how pollinators can maximize their income, not view of the plant (KI, Song 2014)
%%%%%%

\end{itemize}


quantity and quality: Pollen must be from the right plant. Problem for scarce plants. 

%natural-frequency experiment


\subsection*{Research Questions}
In this study I want to address the following questions: 

\begin{enumerate}
	\item Does the per-flower pollinator visitation change with the frequency of the flower in a floral community? \\

	\item	What kind of a frequency dependent relationship can be found? (linear, positive, negative, saturated, hump shaped, sigmoid)\\

	\item	Are there drivers for frequency dependence? (e.g. flower cover, species richness, similarity index, degree of clustering, forager behavior)\\

\end{enumerate}

In this thesis, I tested the existence of FD within a gradient of frequencies of five flowering plant species within a grassland plant community on the area of the Jena Experiment. 
To explore the deeper drivers of FD I developed a spatially explicit Agent-Based Model (ABM). The autonomous agents respond with set foraging behavior rules to changing frequency and clustering values for the two co-flowering plant species. The results of the model were compared to field data.


% in machen papern steht hier schon ziemlich viel zu den Methoden bzw ei die Modelle aufgebaut sind. Mache ich eher nicht so gerne wegen wiederholung
