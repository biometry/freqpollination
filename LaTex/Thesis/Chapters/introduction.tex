\section{Introduction}

The majority of flowering plants species depend on pollination by insects for reproduction. Any advantage in attractiveness can be crucial for the plants fitness in the competition for pollination service of a shared pollinator. However, the foraging behavior is complex and influenced by various factors \citep{goulson1999foraging}. One pattern affecting the flower decision of pollinators is frequency dependence. Generally, frequency dependence of survival or reproduction is defined as relative fitness of a species as a function of its frequency in the community \citep{ayala1974frequency,wright1946genetics}. In the case of plant-pollinator interactions, it occurs if the pollinator is influenced in its foraging behavior by the relative density of a species in a flowering community. Other species are neglected even if they are closer or more rewarding. Depending on their proportion in the flowering community, plants receive additional visits which can increase relative fitness and give an reproductive advantage due to increased seed set. According to ecological theory, frequency dependency can have far-ranging consequences for species coexistence and the maintenance of diversity \citep{levin1972low}. Two types of frequency dependence can be differenced: Negative frequency dependence describes the preference of pollinators for the rare flower types which can result in a stable polymorphic equilibrium and an increase in floral diversity \citep{gigord2001negative}. On the other hand, positive frequency dependence defines an advantage for the common phenotype which tends to reduce diversity in modeling studies (\citealt{may1974stability}, but see \citealt{bever1999dynamics}, \citealt{molofsky2002novel}). For animal-pollinated plant species, optimal foraging theory predicts that under most circumstances pollinators should favor common flower types over rarer ones \citep{kunin1996pollinator}.  \\

While the positive effect of density dependence for pollination success is well studied (e.g. \citealt{essenberg2012explaining,bernhardt2008effects,kunin1993sex,morris2010benefit}), frequency dependence has rarely been tested. However, the few previous studies of frequency dependent pollination cover laboratory, field and modeling experiments. \\
In the review by \cite{smithson2001pollinator}, 11 of 13 lab experiments using artificial flowers on a "bee-board" showed significant results for frequency dependence. 10 of those were done with rewarding flowers and resulted in positive frequency dependence favoring the abundant corolla color \citep{smithson1996frequency,smithson1997density}.  Negative frequency dependence was observed in the only experiment with non-rewarding flowers \citep{smithson1997negative}. \\
Field experiments on frequency dependence were either wholly or partly manipulative and concentrated on color morphisms. \cite{epperson1987frequency} found the rare white morph of \textit{Ipomoea purpurea} to be undervisited (but not the colored morphs) and \cite{gigord2001negative} proved negative frequency dependent selection for the rewardless orchid \textit{Dactylorhiza sambucina}, both supporting the lab experiments. However, \cite{Eckhart2006frequency} was the first to prove negative frequency dependence for a rewarding species (\textit{C. xantiana ssp. xantiana}) and to include natural frequencies. Other studies had no significant results (eg. \citealt{jones1996pollinator, mogford1978pollination}) and experiments on natural flower communities have not been conducted yet to our knowledge.\\
While foraging models are relatively common, few investigate frequency dependence. The game-theoretic model by \cite{kunin1996pollinator} suggests pollinators should favor common flower types over rarer ones when resource availability is high. The similar mathematical model of \cite{song2014adaptive} also concentrates on the pollinator perspective by applying rules of optimal foraging strategy to observe under which conditions the pollinators are able to maximize their net energy intake. Spatial explicit models grew in number over the last years addressing a range of foraging topics \citep{dornhaus2006benefits,bukovac2013bees,faruq2013biological}. Frequency dependent pollination is only subject to the model by \cite{hanoteaux2013effects} who tested survival strategies for less attractive species over multiple generations. \\
In summary, previous research on frequency dependence is limited and inconsistent between lab, field and simulation data. Rewarding flowers are underrepresented in field and lab experiments and studies of natural flower communities lack completely.  Furthermore, direct comparison of model and field data to cross-validate findings were only done for related questions such as density effects \citep{essenberg2012explaining} and the learning abilities of bees \citep{dyer2014bee} but never for frequency dependence.\\

Furthermore, little is known of the influencing factors on frequency dependence which can be responsible for differing results of previous research.  \cite{smithson2001pollinator} hypothesized in her review about factors including sampling size, floral traits and vision distance but did not test them. Density and spatial distribution of flowers can influence the perception of frequency of the pollinator and therefore also have interaction effects on frequency dependence.\\ 
While flower density is known to influence the foraging behavior of pollinators (eg. \citealt{kunin1993sex,essenberg2012explaining}), a possible interaction with frequency dependence was not considered in most cases. Exceptions were \cite{smithson1997density} who observed visitation rates for densities between 5 and 10\% in their lab experiment without any significant result. \cite{kunin1996pollinator} and \cite{song2014adaptive}  included density as factor in their mathematical model and found it strongly influencing the optimal foraging strategy of pollinators.\\

In contrast to habitat fragmentation, the influence of spatial structure and distribution of flowers is not well studied although flowers typically exist in patchy distributions of various levels and sizes. Flowers are often clustered in inflorescences which are again clustered on the plant itself. Also individual flowers are likely to be aggregated in patches over the meadow. Usually, the proportion of flowers visited by pollinators declines with increasing cluster size, probably due to limited memory structure and the avoidance of previously visited flowers \citep{goulson2000pollinators}. \cite{geslin2014effect} found the foraging behavior of bumble bees (\textit{Bombus terrestris}) affected by the spatial distribution of two co-flowering species in a controlled lab experiment. Again, the only study about spatial distribution of flowers in the context of frequency dependence was done by \cite{hanoteaux2013effects}. Within their model four levels of flower aggregation significantly influenced the survival rate of the less attractive species. The highest survival rates were found for big clusters in low frequencies and no clusters in high frequencies.\\ 

Given the general low quantity of studies concerning frequency dependent pollination and their inconsistent results, I want to address the following questions in this thesis:

\begin{enumerate}
	\item Does frequency dependent pollinator foraging exist for rewarding species in natural floral communities?
	\item	What kind of a frequency dependent relationship can be found?
	\item	What are important factors influencing frequency dependence?
\end{enumerate}

In a initial field study, I collected data on per-flower visitation rates of five different flowering rewarding plant species. Observations were made over a range of frequencies in their natural grassland plant communities in the area of the Jena Experiment. To understand which factors are influencing frequency dependence, I developed a spatially explicit model of two rewarding co-flowering plant species sharing pollination services. Agent-based models ("ABM", also known as individual-based models "IBM") are a valuable tool for assessing interactions in dynamic networks like financial markets, game theory, spread of diseases or, like in this case, ecosystems \citep{deangelis2005individual}. The model contains multiple agents which behave independently after given behavior rules and are able to interact with the environment and each other. Agent-based models are especially suitable for analyzing behavior shifts with changing environmental conditions. In the model, frequency, floral cover and cluster size were included in the main analysis to broaden the knowledge gained by the exploratory field study. This approach makes it possible to identify subsequent research options to further evaluate frequency dependence.

%The combination of an ABM with experimental data is a rare but promising approach. Fist to apply this method in foraging models was \citet{dyer2014bee}. They trained honey bees in a lab experiment to fine color discrimination to check for their flexibility to change when the reward changes between flower types. Afterwards, \citet{dyer2014bee} confirmed the findings with a ABM. 


%NOTES:
%
%+FDP
%- fixation on one phenotype for color morphs of one species
%- possible reduction of diversity (rare flowers do not get enough pollination, reproduction disadvantage) see Molofsky and Bever 2002
%- the most frequent species is becoming even more frequent
%
%reasons/explanations:
%- search image hypothesis (sensory system becomes trained, minimize search times)
%- search rate hypothesis (trade-off search time and probability to find the next rewarding flower)
%
%-FDS
%- promotes phenotye diversity
%- can promote landscape diversity
%- known for non-rewarding flowers
%
%reasons:
%- if a species is rare enough, the negative experience is not stored long enough in the short term memory and it gets exploratory vistits by the pollinator
%- naive pollinator hypothesis: non-rewarding species get visits from naive pollinators without knowledge to distinguish between rewarding and unrewarding species
%
%%%%%



%description other ABMs:
%\citet{dornhaus2006benefits} looked at the benefits of a recruitment system and colony sizes. \citet{faruq2013biological} compared the foraging success while applying different flower colors by varying the wavelengths over time. \citet{bukovac2013bees} simulated the difference between the parallel visual scan of honey bees and the serial visual scan of bumble bees to for the ability to avoid distractions during foraging. 