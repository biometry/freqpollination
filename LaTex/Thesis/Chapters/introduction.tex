\section{Introduction}


\subsection*{Broad Topic}

Frequency-dependence (FD) of survival or reproduction occurs when the relative fitness of a species changes as a function of its frequency in the community. According to ecological theory, frequency dependent pollination could have far-ranging consequences for species coexistence and the maintenance of diversity: While negative frequency dependence (i.e., preferential pollination of rare flower types) is thought to increase diversity, positive frequency dependence (higher pollination success for common flower types) tends to reduce diversity in modeling studies (\citealt{may1974stability}, but see \citealt{bever1999dynamics}, \citealt{molofsky2002novel}).
For animal-pollinated plant species, optimal foraging theory predicts that under most circumstances pollinators should favor common flower types over rarer ones \citep{kunin1996pollinator}, but so far this has rarely been tested. \\


Science agrees on a form of density dependence for pollination success from various studies with lab, field and modelling approaches (usually positive):

\citep{bernhardt2008effects} field
\citep{elliott2009effects} field
\citep{essenberg2012explaining} numerical Model and field
\citep{Kunin1997} field
\citep{kunin1993sex} experimental
\citep{morris2010benefit} model, field
\citep{rands2010effects} model
\citep{stout1998influence} lab

....noch was zu cluster (literatur?)

\subsection*{what has been done before}

Previous studies of frequency-dependent pollination cover laboratory, field and modeling experiments.

In the review by \cite{smithson2001pollinator}, 11 of 13 lab experiments using artificial flowers on a "bee-board" showed significant results for frequency dependence. 10 of those were done with rewarding flowers and resulted in positive frequency dependence\citep{smithson1996frequency,smithson1997density}). The only experiment resulting in negative frequency dependence was done with non-rewarding flowers \citep{smithson1997negative}. 

The few field experiments on frequency dependence are either wholly or partly manipulative and concentrate on color morphisms. \cite{epperson1987frequency} found the rare white morph of \textit{Ipomoea purpurea} to be undervisited (but not the colored morphs) and \cite{gigord2001negative} proved negative frequency-dependent selection in the rewardless orchid \textit{Dactylorhiza sambucina}, both supporting the lab experiments. However, \cite{Eckhart2006frequency} was the first to prove negative frequency dependence for a rewarding species (\textit{C. xantiana ssp. xantiana}) and other studies had no significant results (eg. \citealt{jones1996pollinator, mogford1978pollination}). Experiments on natural flower communities lack completely to our knowledge. 

While foraging models are comparatively common, few investigate frequency dependence. The game-theoretic model by \cite{kunin1996pollinator} suggests pollinators should favor common flower types over rarer ones when resources availability is high. The similar mathematical model of \cite{song2014adaptive} also concentrates on the pollinator perspective (optimal foraging strategy, maximising their net energy intakt) The only spatial explicit model focusing on frequency dependent pollination is \cite{hanoteaux2013effects} who addressed the survival rates of less attractive species over multiple generations. 



\subsection*{Gap in knowledge/Introduce Problem}
problem:
- inconsistent findings between lab/field/model, only suggestions and hypothesis why
- Big lack of research in the field: not manipulated, other than morph, natural frequencies, rewarding
- Lack in model: Model: plant-perspective, spatial explicit (most models see how pollinators can maximize their income, not view of the plant (KI, Song 2014))
- never been a comparison model and field data for fd

Also:
- combination of cover and cluster not yet known, never both observed (cluster only once adressed: Hanoteux; cover adressed in smithson1997density but only 5-10\%, Kunin1996/Song2014, Gigord,Epperson\& Eckard no density, many density-studies but few frequency-density )
cluster (nofreq): Goulson2000, Geslin1014, 

+FDP
- fixation on one phenotype for color morphs of one species
- possible reduction of diversity (rare flowers do not get enough pollination, reproduction disadvantage) see Molofsky and Bever 2002
- the most frequent species is becoming even more frequent

reasons/explanations:
- search image hypothesis (sensory system becomes trained, minimize search times)
- search rate hypothesis (trade-off search time and probability to find the next rewarding flower)

-FDS
- promotes phenotye diversity
- can promote landscape diversity
- known for non-rewarding flowers

reasons:
- if a species is rare enough, the negative experience is not stored long enough in the short term memory and it gets exploratory vistits by the pollinator
- naive pollinator hypothesis: non-rewarding species get visits from naive pollinators without knowledge to distinguish between rewarding and unrewarding species

%%%%


quantity and quality: Pollen must be from the right plant. Problem for scarce plants. 



\subsection*{My approach /What method data did I use?/Questions}
In this thesis I want to address the following questions: 

\begin{enumerate}
	\item Does frequency dependence selection of pollinators also exist for rewarding species in natural floral communities? \\
	\item	What kind of a frequency-dependent relationship can be found?\\
	\item	What are important factors influencing frequency dependence?\\
\end{enumerate}

I observed the per-flower visitation rates to five different flowering plant species within natural grassland plant communities on the area of the Jena Experiment. Observations were made over a range of natural frequencies, floral cover and species richness. To explore the important factors influencing frequency dependence I developed a spatially explicit agent-based model (ABM). The autonomous agents respond with set foraging behavior rules to changing frequency, cover and clustering values for two rewarding co-flowering plant species sharing pollination services. 
Subsequently, the results of the model were compared to field data to understand underlying rules for FD. 

