\section{Introduction}


\subsection*{Frequency-dependence}

Frequency-dependence (FD) of survival or reproduction is defined as relative fitness of a species as a function of its frequency in the community \citep{ayala1974frequency,wright1946genetics}.
According to ecological theory, frequency dependency could have far-ranging consequences for species coexistence and the maintenance of diversity. 
Negative frequency dependence occurs if the fitness of a phenotype increases if it is rare and is thought to increase diversity. In the case of plant-pollinator-interactions, negative FD describes the preference of pollinators for the rare flower types which can result in a stable polymorphic equilibrium and an increase in floral diversity. Positive frequency dependence on the other hand describes in an increased fitness for common phenotype. Hence, a higher pollination success for common flower types which tends to reduce diversity in modeling studies (\citealt{may1974stability}, but see \citealt{bever1999dynamics}, \citealt{molofsky2002novel}).
 For animal-pollinated plant species, optimal foraging theory predicts that under most circumstances pollinators should favor common flower types over rarer ones \citep{kunin1996pollinator}. While the positive effect of density dependence for pollination success is well studied (e.g. \citealt{essenberg2012explaining,bernhardt2008effects,kunin1993sex,morris2010benefit}, frequency dependence has rarely been tested.

\subsection*{Literature}

Previous studies of frequency-dependent pollination cover laboratory, field and modeling experiments.

In the review by \cite{smithson2001pollinator}, 11 of 13 lab experiments using artificial flowers on a "bee-board" showed significant results for frequency dependence. 10 of those were done with rewarding flowers and resulted in positive frequency dependence\citep{smithson1996frequency,smithson1997density}). The only experiment resulting in negative frequency dependence was done with non-rewarding flowers \citep{smithson1997negative}. 

The few field experiments on frequency dependence are either wholly or partly manipulative and concentrate on color morphisms. \cite{epperson1987frequency} found the rare white morph of \textit{Ipomoea purpurea} to be undervisited (but not the colored morphs) and \cite{gigord2001negative} proved negative frequency-dependent selection in the rewardless orchid \textit{Dactylorhiza sambucina}, both supporting the lab experiments. However, \cite{Eckhart2006frequency} was the first to prove negative frequency dependence for a rewarding species (\textit{C. xantiana ssp. xantiana}) and other studies had no significant results (eg. \citealt{jones1996pollinator, mogford1978pollination}). Experiments on natural flower communities lack completely to our knowledge. 

While foraging models are comparatively common, few investigate frequency dependence. The game-theoretic model by \cite{kunin1996pollinator} suggests pollinators should favor common flower types over rarer ones when resources availability is high. The similar mathematical model of \cite{song2014adaptive} also concentrates on the pollinator perspective by applying rules of optimal foraging strategy and observe under which conditions the pollinators are able to maximize their net energy intake. The only spatial explicit model focusing on frequency dependent pollination is \cite{hanoteaux2013effects} who addressed the survival rates of less attractive species over multiple generations. 

\subsection*{Gap in knowledge/ Problem}

Previous research on FD is scarce and inconsistent between lab, field and simulation data.
All field experiments were partly or fully manipulated studies on color morphs. \cite{Eckhart2006frequency} was the first to study a rewarding species and to include natural frequencies. Still, rewarding flowers are underrepresented and studies of natural flower communities lack completely.  Furthermore, direct comparison of model and field data to cross-validate findings were only done for related questions such as density effects \citep{essenberg2012explaining} and the learning abilities of bees \citep{dyer2014bee} but never for FD.  

Next to the yet not fully proven existence of FD are its influencing factors which can be responsible for  differing results of previous research. \cite{smithson2001pollinator} hypothesized in her review about possible reasons but until now, no study was conducted to fill the knowledge gap. 

Floral cover is known to influence the foraging behavior of pollinators (eg. \citealt{kunin1993sex,essenberg2012explaining}). However, a possible interaction with  frequency dependence was not considered in most cases. Exceptions were \cite{smithson1997density} who observed visitation rates for densities between 5 an 10\% in their lab experiment without any significant result and \cite{kunin1996pollinator} and \cite{song2014adaptive} who included density as factor in their mathematical model. Field experiments generally lack cover analysis. 

The influence of spatial structure and distribution of flowers is not well studies (but:habitat fragmentation) although flowers typically exist in patchy distributions of various sizes. Usually, the proportion of flowers visited by bees decline with increasing cluster size, probably due to limited memory structure and the avoidance of previously visited flowers \citep{goulson2000pollinators}. \cite{geslin2014effect} also found the foraging behavior of bumble bees (\textit{Bombus terrestris}) affected by the spatial distribution of two co-flowering species in a controlled lab experiment. The only study about spatial distribution of flowers in the context of frequency dependence was done by \cite{hanoteaux2013effects}. Their spatially explicit model included four levels of flower agglomeration which significantly influenced the survival rate of the less attractive species (best survival rates: high cluster for low frequencies, low cluster for high frequencies). 

% quantity and quality: Pollen must be from the right plant. Problem for scarce plants. 
%On the other hand, foraging models usually take a pollinator position, maximizing their intake \cite{kunin1996pollinator,faruq2013biological,song2014adaptive}.

\subsection*{My approach/Questions}
In this thesis I want to address the following questions: 

\begin{enumerate}
	\item Does frequency dependence selection of pollinators also exist for rewarding species in natural floral communities? \\
	\item	What kind of a frequency-dependent relationship can be found?\\
	\item	What are important factors influencing frequency dependence?\\
\end{enumerate}

I observed the per-flower visitation rates to five different flowering plant species within natural grassland plant communities on the area of the Jena Experiment. Observations were made over a range of natural frequencies, floral cover and species richness. To explore the important factors influencing frequency dependence I developed a spatially explicit agent-based model (ABM). The autonomous agents respond with set foraging behavior rules to changing frequency, cover and clustering values for two rewarding co-flowering plant species sharing pollination services. Subsequently, the results of the model were compared to field data to understand underlying rules for FD. 




%
%NOTES:
%
%+FDP
%- fixation on one phenotype for color morphs of one species
%- possible reduction of diversity (rare flowers do not get enough pollination, reproduction disadvantage) see Molofsky and Bever 2002
%- the most frequent species is becoming even more frequent
%
%reasons/explanations:
%- search image hypothesis (sensory system becomes trained, minimize search times)
%- search rate hypothesis (trade-off search time and probability to find the next rewarding flower)
%
%-FDS
%- promotes phenotye diversity
%- can promote landscape diversity
%- known for non-rewarding flowers
%
%reasons:
%- if a species is rare enough, the negative experience is not stored long enough in the short term memory and it gets exploratory vistits by the pollinator
%- naive pollinator hypothesis: non-rewarding species get visits from naive pollinators without knowledge to distinguish between rewarding and unrewarding species
%
%%%%%
