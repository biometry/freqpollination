\section{Introduction}

\subsection{Introduction (Storyline)}

\begin{itemize}

\item pollination crucial for most plants (self-incompatible)
\item Interspecific and intraspecific competition
\item pollinator service shared ressource of coflowering plants
\item pollinator attractiveness
\item increased forager activity and attractiveness can respond in higher fitness

\item response of pollinators to certain variables like density, frequency, traits, etc 
\item complex multidimensional interactions
\item detects bits and peaces to get a greater understanding
\item The effect on density is rather well studied
\end{itemize}


From the proposal:\\
Frequency-dependence (FD) of survival or reproduction occurs when the relative fitness of a species changes as a function of its frequency in the community. For animal-pollinated plant species, optimal foraging theory predicts that under most circumstances pollinators should favor common flower types over rarer ones \citep{kunin1996pollinator} , but so far this has rarely been tested. Most previous studies of frequency-dependent pollination focused on within-species variation in pollination success of different flower color morphs (e.g. \citealt{Eckhart2006frequency} , \citealt{smithson2001pollinator }). Other floral traits (e.g. morphology and scent) and interspecific differences in pollination success have rarely been considered in the context of frequency dependence.

According to ecological theory, frequency dependent pollination could have far-ranging consequences for species coexistence and the maintenance of diversity: While negative frequency dependence (i.e., preferential pollination of rare flower types) is thought to increase diversity, positive frequency dependence (higher pollination success for common flower types) tends to reduce diversity in modeling studies (May 1974, but see \citealt{bever1999dynamics}, \citealt{molofsky2002novel}).


\subsection{Research Questions}

1.Does the per-flower pollinator visitation change with the frequency of the flower in a floral community? \\

2.What kind of a frequency dependent relationship can be found? (linear, positive, negative, saturated, hump shaped, sigmoid)\\

3.Are there drivers for frequency dependence? (e.g. flower cover, species richness, similarity index, degree of clustering, forager behavior)\\


In this thesis, I tested the existence of FD within a gradient of frequencies of five flowering plant species in the area of the Jena Experiment. Furthermore, I used an spatially explicit Agent-Based Model (ABM) for the modeling approach. The autonomous agents respond with set foraging behavior rules to changing frequency and clustering values for the two co-flowering plant species. 

The results of the model were compared to field data.

