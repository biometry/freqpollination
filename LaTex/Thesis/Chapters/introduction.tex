\section{Introduction}


\subsection*{Broad Topic}

Frequency-dependence (FD) of survival or reproduction occurs when the relative fitness of a species changes as a function of its frequency in the community(Wright 1948). For animal-pollinated plant species, optimal foraging theory predicts that under most circumstances pollinators should favor common flower types over rarer ones \citep{kunin1996pollinator} , but so far this has rarely been tested. 

According to ecological theory, frequency dependent pollination could have far-ranging consequences for species coexistence and the maintenance of diversity: While negative frequency dependence (i.e., preferential pollination of rare flower types) is thought to increase diversity, positive frequency dependence (higher pollination success for common flower types) tends to reduce diversity in modeling studies (May 1974, but see \citealt{bever1999dynamics}, \citealt{molofsky2002novel}).

%%%

+FDP
- fixation on one phenotype for color morphs of one species
- possible reduction of diversity (rare flowers do not get enough pollination, reproduction disadvantage) see Molofsky and Bever 2002
- the most frequent species is becoming even more frequent

reasons/explanations:
- search image hypothesis (sensory system becomes trained, minimize search times)
- search rate hypothesis (trade-off search time and probability to find the next rewarding flower)

-FDS
- promotes phenotye diversity
- can promote landscape diversity
- known for non-rewarding flowers

reasons:
- if a species is rare enough, the negative experience is not stored long enough in the short term memory and it gets exploratory vistits by the pollinator
- naive pollinator hypothesis: non-rewarding species get visits from naive pollinators without knowledge to distinguish between rewarding and unrewarding species

%%%%

-biodiversity loss /important to maintain and understand 
-what are the divers for biodiversity?
-pollinator loss, have to understand underlying rules/drivers/
-pollination crucial for most plants (self-incompatible)
-Interspecific and intraspecific competition
-pollinator service shared ressource of coflowering plants
-increased forager activity and attractiveness can respond in higher fitness
-response of pollinators to certain variables like density, frequency, traits got a lot of attention over the last 2 decades since pollinator decline
-complex multidimensional interactions
-other factors: presence/absence of similar flowering species, spatial scale
-detects bits and peaces to get a greater understanding
-The effect on density is rather well studied


Science agrees on a form of density dependence for pollination success from various studies with lab, field and modelling approaches:
	
	\citep{bernhardt2008effects} field
	\citep{elliott2009effects} field
	\citep{essenberg2012explaining} numerical Model and field
	\citep{Kunin1997} field
	\citep{kunin1993sex} experimental
	\citep{morris2010benefit} model, field
	\citep{rands2010effects} model
	\citep{stout1998influence} lab
	
usually positive
	
%%%%%%%%%%%%%%
	
quantity and quality: Pollen must be from the right plant. Problem for scarce plants. 

\subsection*{what has been done before}

- colored discs are artificial flowers arranged in patterns/randomly
- only one experiment ever with variable reward (Real 1990, buch leider nicht da), angeblich (nach Smithson 2001 signifikant +FDP)

Most previous studies of frequency-dependent pollination focused on within-species variation in pollination success of different flower color morphs in the field or in the lab with a "bee-board" with corolla color as differing flower trait. 

In the review by \cite{smithson2001pollinator}, 11 of 13 lab experiments using artificial flowers on a "bee-board" with rewarding flowers showed significant results for frequency dependence. 10 of those were done with rewarding flowers and resulted in positive frequency dependence. The only experiment resulting in negative frequency dependence was done with non-rewarding flowers \citep{smithson1997negative}. The few field experiments on frequency dependence concentrate on color morphs and are not consistent. While \cite{epperson1987frequency} found the rare white morph of \textit{Ipomoea purpurea} to be undervisited (but not the colored morphs), \cite{Eckhart2006frequency} was the first to prove negative frequency dependence for a rewarding species (\textit{C. xantiana ssp. xantiana}). Other experiments had no significant results (eg. \citealt{jones1996pollinator}) and experiments on natural flower communities lack completely to our knowledge. 


\subsection*{Gap in knowledge/Introduce Problem}

rarely been addressed: 
- FD in the field
- For species within a floral community (in contrast to flower morphs)
- Model: hanoteux (but: survival/generations) and K and I (but:best strategy, bee-perspective)
-

my approach new because: 
-never been a comparison model and field data for fd
-non-linear relationship (consistent)
-findings in the field
- most models see how pollinators can maximize their income, not view of the plant (KI, Song 2014)


\subsection*{My approach /What method data did I use?/Questions}
In this thesis I want to address the following questions: 

\begin{enumerate}
	\item Does the per-flower pollinator visitation change with the frequency of the flower in a natural floral community? \\
	\item	What kind of a frequency dependent relationship can be found?\\
	\item	What are important factors influencing frequency dependence?\\
\end{enumerate}

I observed the visitation rates to five different flowering plant species within a grassland plant community on the area of the Jena Experiment for a range of natural frequencies. To explore the deeper drivers of FD I developed a spatially explicit agent-based model (ABM). The autonomous agents respond with set foraging behavior rules to changing frequency and clustering values for two co-flowering plant species. The results of the model were compared to field data to understand underlying rules for FD. 

been considered in the context of frequency dependence.


