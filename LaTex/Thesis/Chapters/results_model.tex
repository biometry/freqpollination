\label{ch:results_model}

\subsection{Results}

\subsubsection*{Per-flower Visitation Rate}

\begin{figure} [!ht] %results:PFV
	\centering
	\includegraphics[width=16cm]{Images/PFV}
	\caption{The per-flower visitation rate shows a frequency dependence with a cubic relationship. The same data is plotted relative to per-flower visitation of the other species to visualize the relative fitness and intensity of frequency dependence (e-h). A value of one means no benefit in visits per flower for either species. The frequency effect is stronger for higher floral cover while an increase in cluster size reduces the frequency dependence. The peak at 90\% for 10\% cover in a 100 flowers per cluster environment is due to a statistical outlier highly influencing the mean. The parameter combination was run again for validation (supplementary material, Fig.~\ref{fig:outlier}). Note that the statistical variability increases for very low frequencies in highly clustered scenarios (grey error bars). If a rare species is occurring only in one cluster on the meadow, it will receive either no or multiple visits. See Fig.~\ref{fig:PFVgross} in the supplementary material for an enlarged image.}
	\label{fig:PFV}
\end{figure}


The per-flower visitation shows for small clusters a similar cubic function as the data collected in the Jena Experiment (Fig.~\ref{fig:PFV}a,b). Within the first 20\% there is a steep increase in visits per flower. Afterwards, since for all cover values above 5\% the additional gain of visits is not proportional to the increase of flowers due to higher frequency, the per-flower visitation drops with a minimum around 80\%. Close to 100\%, when the species becomes dominant, the per-flower visitation rises again. Cover and cluster size both influence the frequency dependence. The higher the cover, the lower the per-flower visitation and the bigger the clusters the less visible is the frequency dependence. Simulations with more than 10 flowers per cluster show a high variance for frequencies below 10\% and very low to no frequency dependence afterwards (Fig.~\ref{fig:PFV}c,d). 

The same data is plotted as relative visitation rate to see which species has a higher fitness and to take the variance in the sum of visits into account (Fig.~\ref{fig:PFV}e-h; see section "Global visitation"). If the value is one, both species receive the same per-flower visitation. For 5\% cover, the bee-agent exhibits a positive frequency dependence. If the cover is higher, the rare species benefits from a negative frequency dependence. Cluster size reduces the effect and the data approach a slightly negative frequency dependence (Fig.~\ref{fig:PFV}h). 

\subsubsection*{Pollination Success}

\begin{figure} [!ht] %results:POC
	\centering
	\includegraphics[width=15cm]{Images/POC}
	\caption{Frequency dependence of proportion of successful pollinator visits for different combinations of floral cover and cluster size. The pollen carryover value defines the maximum number of heterospecific visits within a successful pollination is possible. With a pollen-carryover rate of one, the pollen can only be carried to the next flower. Therefore, the ratio of successful pollinations per visit can be seen as indicator for flower constancy \citep{montgomery2009pollen}. A high pollen-carryover rate is only important for a low cover and no-cluster environment. With increasing cover and cluster, the ratio becomes steeper for low frequencies which stands for more qualitative visits.}
	\label{fig:POC}
\end{figure}

The degree of pollen carryover is defined as maximum number of heterospecific visits within a successful pollination can take place. In the model, I tested values from 1 (strong heterospecific pollen interference) to 16 (weak heterospecific pollen interference). Figure~\ref{fig:POC} gives the proportion of all visits where a successful pollination took place. The first 20\% frequency are crucial for all parameter-value combination. A very steep increase up to 80\% successful visits is followed by a moderate linear increase up to 100\% for exclusive existence. The degree of pollen carryover only makes a difference for small cover and cluster values (Fig.~\ref{fig:POC}a). The higher the cover and the bigger the clusters, the better is also the proportion of successful pollination, even for small frequencies, independent of the pollen-carryover rate(Fig.~\ref{fig:POC}c,f,g-i). 

\subsubsection*{Global visitation}

\begin{figure} [!ht] %results:SUM
	\centering
	\includegraphics[width=16cm]{Images/SUM}
	\caption{Summed visits to both species show a frequency dependence for low cluster values. Depending on the floral cover it is quadratic or a fourth-degree polynomial relationship. The maximum number of visits (maximal efficiency) is achieved for highly unequal or balanced frequencies of the two plant species, depending on floral cover and cluster size.}
	\label{fig:SUM}
\end{figure}

The sum of all visits to both plants together changes with the ratio of their abundance (Fig.~\ref{fig:SUM}). A strong cover-dependent pattern is visible for small cluster sizes. The total visits have an u-shaped relationship with frequency for 5\% cover and a shape similar to a fourth-degree polynomial function for higher cover values. The visitation drops to a minimum at 90:10 ratio and peaks again for balanced frequencies. Cluster size reduces the frequency dependence. \\
Note that the mean number of total visits varies in addition to the shape for different values of cover and cluster size. Both parameters have a frequency independent influence on the visitation rate (Supplementary material, Fig.~\ref{fig:nonfreq}). Floral cover shows a saturated curve and the visits for degree of clustering have a hump-shaped relationship with a peak at an intermediate aggregation level of 5-10 flowers per cluster.

\subsection*{Sensitivity Analysis}
Aim of the sensitivity analysis is to understand influence of behavioral rules on the outcome of the model. Therefore, pollinator density, reward regrowth, search time and vision were tested for a (unnaturally) broad range of values set in comparison to the empirically founded default values. Vision, search time and the number of bees on the meadow influence the sum of visits (Fig.~\ref{fig:SA_SUM}). A larger range of vision leads to more visits, the search limit reduces the number of visits and more bees lead again to more visits per time unit. The reward function has an small negative influence on the total number of visits for a very high regrowth rate.

Furthermore, a very high replenishment rate (0.1J/sec, yellow line in Fig.~\ref{fig:SA_reward}) has a reversing effect on the frequency dependence: Rare species receive disproportionally few visits whereas common species benefit from a positive frequency dependence. The influence of frequency is less severe with increasing cluster size. 

Every bee-agent can detect flowers in a 180$^{\circ}$ cone-shaped array of patches in front of them. The number of patches in that array is determined by the vision distance. A large range of vision increases the frequency dependence even in a heavily clustered model environment (Fig.~\ref{fig:SA_view}). If the bee-agents are only able to see the direct neighbor, the frequency dependence is reversed to favor the common species for low cluster values. 

If a bee-agent searches longer than a given search time limit unsuccessfully for a unvisited and preferred flower, the probability to switch preferences will increase by 10\% with every additional step. The search limit was altered from 1 to 50 seconds in the sensitivity analysis. The results are similar to the effect of vision as they also change the probability to switch preferences. Higher search time limits lead to stronger negative frequency dependence benefiting the rare species. A search limit of 1 second reduces the dependency (Fig.~\ref{fig:SA_flight}).

Beside the expected increase of absolute visits, a change of pollinator density has no effect on the outcome of the model at any cover or cluster values (Fig.~\ref{fig:SA_bees}).


%%%REWARD
% Hence, a high reward can fix decrease diversity
% with very high reward rates, the bees find rarely a flower with bad reward and are changing preference less often. The only other way of changing preference is unsuccessful flight and that favors the common species
%IMPORTANT! Explained all lab experiments dome by smithson et al 1997a,1997b & 1996

%%VIEW
%with high vision bee-agents are able to detect flowers of the rare species more quickly and are less likely to change their preference
%similar result as changing search time limit or make density higher
%if the bee can see further it will fly more directly to the next flower without too much random walk. but on the other hand it will fly for longer distances instead of changing preference to the more abundant species. Less efficient, less total visits

%%%FLIGHT
%with higher search time limit, the bee continues to search for the rare species instead of changing to the more abundant species. That is less efficient (less absolut visits) and enhances the FD