\documentclass{article}
\title {The effects of flower frequency dependence for pollination in a similarity and spatial cluster gradient (working copy)}
\author {Helen Czioska}


%\usepackage{hyperref}
%\usepackage{color}
%\usepackage{xcolor}
%\usepackage{graphicx}
%\usepackage{multicol}
%\usepackage{multirow}
%\usepackage{booktabs}



\begin{document}
\maketitle

\section{Abstract}

\section{Introduction}

\section{Methods}

\subsection{Data Collection}

The Jena Experiment:

\begin{itemize}
\item	50°55´N,11°35´E; 130 m a.s.l.
\item	Established in 2002
\item	10 hectars
\item	Arable field for 40 years before experiment started (therefore strongly fertilized)
\item	Plots are mowed every June and September
\item	Main experiment has 82 plots, each 20x20m (400m²)
\item	Originally sown species mix of 1,2,4,8,16 or 60 species, divided into four blocks (randomized complete block design) along abiotic gradients (mainly soil sand content)
\item	Part of the Plot is weeded twice a year (not my area)
\item	I collected data in the “old invasion plots” (6.35m x 4.55m , 28,98m²) and in the “new invasion plots” (5.35m x 3.55m, 19m²)
\item	Observations were made only during good weather conditions (max partly overcast, no rain, max light wind, min. 15 degree)
\item	Sampling time between 9am and 5pm (there was normally heavy fog and moist in the mornings so I could only start sampling from 10 or even 11)
\item	Sampling occurred between 20.7. -  12.8.
\end{itemize}

\subsubsection{Choosing Species and Plots}

I chose 5 species to observe (Those species were chosen because they were present in min. 5 plots with a differing frequency):

% Table generated by Excel2LaTeX from sheet 'Spec'
\begin{table}[htbp]
  \centering
  \caption{Tabel 1: Focal species}
    \begin{tabular}{rrrrrr}
    \toprule
    \textbf{Short} & \textbf{Name} & \textbf{German Name} & \textbf{Order} & \textbf{Family} & \textbf{Color} \\
    \midrule
    Ono   & Onobrychis viciifolia & Saat-Esparsette & Fabales & Fabaceae & pink+white \\
    Lat   & Lathyrus pratensis & Wiesen-Platterbse & Fabales & Fabaceae & Yellow \\
    Lot   & Lotus corniculatus & Gewöhnliche Hornklee & Fabales & Fabaceae & Yellow \\
    Ger   & Geranium pratense & Wiesen-Storchschnabel & Geraniales & Geraniaceae & Purple \\
    TP    & Trifolium pratense & Wiesen-Klee & Fabales & Fabaceae & Purple \\
    \bottomrule
    \end{tabular}%
  %\label{tab:addlabel}%
\end{table}%


Because the vegetation changed very quickly (heavy rain and very warm temperatures alternating) I chose max. 7 plots (= 14h) to observe at a time. 
Every time I finished a session I did a new sampling of all 82 plots of the Jena Experiment to check for suitable plots with focal plant species and their frequencies for the next round. Those observations were randomly distributed over the next days to prevent time dependencies ( observation times over the whole day for each plot)

\subsubsection{The Sampling}

\begin{itemize}
\item The pollinators were divided into bees, bumblebees, hoverflies and “other”
\item sampled eight of the 80x80cm patches per plot to get 2h of data per frequency and get a good mean over the plot (flowers were often in clusters and not distributed over the whole plot)
\item Patches were distributed in the plot as following:
\end{itemize}

(here should be a graphic...)

\begin{figure}
\includegraphics[scale=4]{plot-design}
\end{figure}


\begin{itemize}
\item When the flowers were very unevenly distributed over the Plot (which happened especially at low frequencies) I chose to observe some patches which contained flowers of the chosen species twice
\item I normally observed two species at once to save time/get a larger dataset. If there were not too many flowers that was easy feasible. If there were Plots with unevenly distributed flowers as explained above I observed the regular 1-8 patches for the evenly distributed species and additionally doubling patches for the uneven species. 
\item Eg. Geranium was flowering at all 8 patches, Lotus only in the southern part (patch 1,2,4,5, and 7). I regularly observed all patches 1-8 for Geranium. Because I was missing 3 patches for Lotus I doubled 1, 7 and 5. During this doubling I still kept track of visitors of the Geranium flowers. So in the end my dataset was the following:
\begin{itemize}
	\item 8+3 Geranium observations
	\item 8 Lotus observations
\end{itemize}
\end{itemize}

\\
\\


Additionally to the data on flower visitation of the focal species I recorded the following data:
\begin{itemize}
\item Flower visits to other flowers except the focal species in the patch
\item Species Richness in the Patch (with names)
\item Species Richness in the Plot (with names and quantities)
\item Floral Cover in Patch and Plot
\item Frequency of the focal species in Patch and Plot
\item Count of individual flowers respectively inflorescence of the focal species
\item PlotID
\item PatchID
\item Date/Time
\end{itemize}



\section{Results}

\end{document}